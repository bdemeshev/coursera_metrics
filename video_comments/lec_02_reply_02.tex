\documentclass[12pt,a4paper]{article}
\usepackage[utf8]{inputenc}
\usepackage[russian]{babel}

\usepackage{amsmath}
\usepackage{amsfonts}
\usepackage{amssymb}
\usepackage{url}
\usepackage[left=2cm,right=2cm,top=2cm,bottom=2cm]{geometry}
\begin{document}


Глава 2.

Правильные названия фрагментов я указывал, но похоже не акцентировал, поэтому здесь явно пишу:

2-1-1 изменить название фрагмента на <<Условное математическое ожидание. Определение>>

\url{http://www.youtube.com/watch?v=XnI9hsL76vo&feature=youtu.be}

2-1-2 Пример подсчета условного математического ожидания

\url{http://www.youtube.com/watch?v=zG9YitS48FE&feature=youtu.be}

ok

2-1-3 Пример подсчета условной дисперсии

\url{http://www.youtube.com/watch?v=TELgNS6QJ7k&feature=youtu.be}

не исправлено старое:

Сначала (в 1:19) появляется синим только 

$E(as + b|r) = aE(s|r) + b$

в 1:31 дополнительно появляется синим 

$E(E(s|r)) = E(s)$

(сейчас эти два пункта появляются не в том порядке, что я их озвучиваю)



2:23 поднять 2 в верхний индекс в конце первой формулы, должно быть

$Var(s)=E(s^2)-(E(s))^2$


2-1-4 исправить название фрагмента на <<Геометрическая иллюстрация условного математического ожидания>>

\url{http://www.youtube.com/watch?v=HXAVgW63m40&feature=youtu.be}

звук стал рассинхронизирован с видео, раньше было ок! нужно обратно его синхронизировать

2-1-5 исправить название фрагмента на <<Условная дисперсия МНК оценок>>

\url{http://www.youtube.com/watch?v=LfETeyOy2C0&feature=youtu.be}


2-1-6  исправить название фрагмента на <<Условная дисперсия МНК оценок. Доказательство>>

\url{http://www.youtube.com/watch?v=ZCnLEhA3Am0&feature=youtu.be}

2-1-7 Дисперсии оценок коэффициентов в общем виде

\url{http://www.youtube.com/watch?v=Ip2t9NPclJA&feature=youtu.be}

ok

2-1-8 исправить название фрагмента  на  <<Доказательство формулы для ковариационной матрицы>>

\url{http://www.youtube.com/watch?v=AGS4FzOF1aU&feature=youtu.be}

 

2-1-9 Оценка ковариационной матрицы %  (длительность 3:45) 

\url{http://www.youtube.com/watch?v=hbtBvOPNhNA&feature=youtu.be}

2:21 исправить формулу на

$se(\hat{\beta}_j)=\sqrt{\widehat{Var}(\hat{\beta}_j | X)}$

2-1-10 Статистические свойства оценок коэффициентов (15 минут)

\url{http://www.youtube.com/watch?v=6_AesNtSnvQ&feature=youtu.be}

4:48 кусок текста <<Векторы отдельных наблюдений...>> пока не надо показывать,

нужно показать только кусок текста <<Синонимы в матричном...>> 

5:03 нужно показать кусок текста <<Векторы отдельных наблюдений...>> (ниже куска <<Синонимы...>>)

\newpage
2-1-11 Построение доверительных интервалов и проверка гипотез (15 минут)

\url{http://www.youtube.com/watch?v=O3sJEu6A9Bk&feature=youtu.be}

0:30 добавляем надпись:

Свойства МНК оценок:

* Успокоительные: оценки несмещенные, состоятельные...

0:54 ниже появляется второй пункт :

* Практические: оценки имеют определенный закон распределения, что позволяет строить доверительные интервалы

0:57 - 0:58 новых надписей с формулами не появляется,
по прежнему висят <<Свойства МНК оценок>> и два упомянутых пункта

1:26 <<Свойства МНК оценок>> с двумя пунктами пропадают, появляется (как сейчас)

Проверять гипотезы можно в двух случаях:

* Число наблюдений велико

* Случайные ошибки нормальны

1:47 стираем предыдущие надписи, появляется

Проверка гипотезы о коэффициенте $\hat{\beta}_j$

* Асимптотически: 

$t=\frac{\hat{\beta}_j-\beta_j}{se(\hat{\beta}_j)} \to N(0,1)$

2:03 появляется ниже 

* При нормальности: 

$t=\frac{\hat{\beta}_j-\beta_j}{se(\hat{\beta}_j)} \sim t_{n-k}$

4:06 вместо странного значка должна быть буква P

$\alpha = P(\text{отвергнуть }H_0 | H_0 \text{ верна})$


в момент 5:53 разбить на две части, вторую часть назвать <<Пример. Доверительный интервал для коэффициента бета>>


 

2-1-12 изменить название на <<Пример. Доверительный интервал для дисперсии>>

\url{http://www.youtube.com/watch?v=GI21j9Co0_A&feature=youtu.be}

 

2-1-13 изменить название на <<Пример. Проверка гипотез о коэффициенте бета>>

\url{http://www.youtube.com/watch?v=OcQCuRegWEI&feature=youtu.be}


6:30 вместо заголовка <<Описание любого теста>> появляется заголовок <<Распространенная форма записи>>

6:57 ниже формулы появляется:

Примерный 95\%-ый доверительный интервал: 

$[\hat{\beta}_j - 2se(\hat{\beta}_j); \hat{\beta}_j + 2se(\hat{\beta}_j)]$

\newpage
2-1-14 Интерпретация стандартной таблички

\url{http://www.youtube.com/watch?v=a_4ayNeZ04k&feature=youtu.be}

ок

2-1-15 Исправить название на <<Особенности проверки гипотез>>

\url{http://www.youtube.com/watch?v=t1etCnF3PyU&feature=youtu.be}

 

2-1-16 Проверка гипотезы о связи коэффициентов и заключение 

\url{http://www.youtube.com/watch?v=f-OVbtCZYvQ&feature=youtu.be}

ok

2-2-1 Работа со случайными величинами в R

\url{http://www.youtube.com/watch?v=PlAbV-oRBgI&feature=youtu.be}

ok

2-2-2  Проверка гипотез о коэффициентах

\url{http://www.youtube.com/watch?v=HUj-5x0zHsc&feature=youtu.be}

ок

2-2-3 Стандартизированные коэффициенты и ложно-значимые регрессоры

\url{http://www.youtube.com/watch?v=kO6PWgrKFlo&feature=youtu.be}

ок

2-2-4 Сохранение и загрузка данных

\url{http://www.youtube.com/watch?v=PIGnOqaYdi8&feature=youtu.be}

ок

2-2-5 Загрузка данных RLMS

\url{http://www.youtube.com/watch?v=zID-2y_GRr8&feature=youtu.be}

\textbf{совсем нет звука} (не считая вступительных аккордов)

\end{document}