\documentclass[12pt,a4paper]{article}
\usepackage[utf8]{inputenc}
\usepackage[russian]{babel}
\usepackage[OT1]{fontenc}
\usepackage{amsmath}
\usepackage{amsfonts}
\usepackage{amssymb}
\usepackage{url}
\usepackage[left=2cm,right=2cm,top=2cm,bottom=2cm]{geometry}
\begin{document}

Лекция 3. Итерация 2.

3-1-1 Прогнозирование во множественной регрессии

\url{http://youtu.be/bJn3Muw-9qk}

4:10 после первой итерации появилась лишняя скобка, она не нужна, правильная формула:

$\hat{y}_i - E(y_i|x_i)$

5:05 эпсилон в последней строке не похож на предыдущие эпсилоны, его надо сделать в общем стиле

5:53 отличающиеся эпсилон. давайте все эпсилон в курсе приведём к одному формату!

правильный эпсилон: $\varepsilon$, все эпсилоны вида $\epsilon$ надо заменить на $\varepsilon$

3-1-2 Пример построения интервалов для прогнозов

\url{http://youtu.be/swWNng-RBho}

ок

3-1-3 Интерпретация коэффициента при логарифмировании

\url{http://youtu.be/Ak7v0Zkt6FE}

ок

3-1-4 Дамми-переменные. Разные зависимости для подвыборок

\url{http://youtu.be/CkR7DC9RpEo}

ок

3-1-5 Проверка гипотезы о нескольких линейных ограничениях

\url{http://youtu.be/thqVoh9Gu1Q}

ок

3-1-6 Пример проверки гипотезы о нескольких линейных ограничениях

\url{http://youtu.be/lV8SyyfZaps}

ок

3-1-7 вывод формулы для гипотезы о незначимости регрессии

\url{http://youtu.be/EVSBiN4lZSk}

ок

3-1-8 Пример проверки гипотезы о незначимости регрессии 

\url{http://youtu.be/xbpA6Dji5mY}

ок

3-1-9 Лишние и пропущенные переменные

\url{http://youtu.be/_UMEXnyxxTo}

ок

3.1.10. Тест Рамсея

\url{http://youtu.be/fuAAh6BZPBo}

ок

3-1-11 Простые показатели качества модели

\url{http://youtu.be/7KwEu1CHFU8}

ок

3-2-1 Графики и переход к логарифмам

\url{http://youtu.be/969UZhaEDcA}

ок

3-2-2 Графики для качественных и количественных переменных 

\url{http://youtu.be/BEfg6DTd_6s}

ок

3-2-3 Оценивание моделей с дамми-переменными. Интерпретация

\url{http://youtu.be/Efkl0iKdCYY}

ок

3-2-4 Построение прогнозов. Доверительный и предиктивный интервал

\url{http://youtu.be/bpauIM3FzPc}

ок

3-2-5 Проверка гипотезы о линейных ограничениях. Графическое представление результатов

\url{http://youtu.be/bGKuiIZjkYg}

ок

3-2-6 Ловушка дамми-переменных. Информационные критерии. Тест Рамсея.

\url{http://youtu.be/BPlPCddiBbg} 

ок

3-2-7 Нано-исследование

\url{http://youtu.be/4DbyGDpNtT8}

0:16 заменить название лекции на <<Проверка сложных гипотез и прогнозирование>>. Текущее название <<мультиколлинеарность>> относится к четвертой лекции.

\end{document}