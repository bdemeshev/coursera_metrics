\documentclass[12pt,a4paper]{article}
\usepackage[utf8]{inputenc}
\usepackage[russian]{babel}
\usepackage{url}
\usepackage{amsmath}
\usepackage{amsfonts}
\usepackage{amssymb}
\usepackage[left=2cm,right=2cm,top=2cm,bottom=2cm]{geometry}
\newcommand{\e}{\varepsilon}
\renewcommand{\b}{\beta}
\newcommand{\hb}{\hat{\b}}
\newcommand{\hs}{\hat{\sigma}}
\usepackage{graphicx}
\begin{document}

Глава 6, итерация 2

6-1-1 Определение автокорреляции 

\url{http://youtu.be/bdTcoQJ0oYA}

0:36 исправить появляющуюся формулу на: (разница в начале формулы):

$Cov(\e_i,\e_j|X)=E(\e_i \e_j|X)=0$ при $i\neq j$

6-1-2 Свойства автокорреляции первого порядка [у доски]

\url{http://youtu.be/X9AEqv7saRo} ок

6-1-3 Последствия автокорреляции

\url{http://youtu.be/NodAGvgYybA}

0:26 исправить оставшуюся ошибку в формуле (в конце перед $u_t$ по-прежнему нет плюса), должно быть:

$\e_{t}=\phi_1 \e_{t-1}+\phi_2 \e_{t-2} +\ldots + \phi_p \e_{t-p}+u_t$

4:15 зарезана скобка в конце строки под корнем, нижняя строка должна быть такой:

В частности, $\widehat{Var}(\hat{\beta}_j|X)=\frac{\hat{\sigma}^2}{RSS_j}$
и $se(\hat{\beta}_j)=\sqrt{\widehat{Var}(\hat{\beta}_j|X)}$

5:58 --- 6:08 слева от формул должны быть <<буллеты>> (точки)

6:48 убрать <<(---)>> перед третьей и четвертой формулами (т.е. слева от каждой формулой  должен быть буллет, а справа --- галка или крест, галки и кресты стоят верно).

7:22 слово <<гипотезы>> убежало вверх строки, третий пункт должен иметь вид:

* Используя обычные $se(\hb_j)$, нельзя строить доверительные интервалы или проверять гипотезы




6-1-4

\url{http://youtu.be/QlVSRkUXKmI}

6-1-5

\url{http://youtu.be/moI2LeRaXe4}

6-2-1Работа с датами в R

\url{http://youtu.be/QZQfP58SdoA} ок

6-2-2 Базовые действия с временными рядами

\url{http://youtu.be/ocoGYXZkQK0} ок

6-2-3 Загрузка данных из внешних источников

\url{http://youtu.be/jr0VmfrZ-Qo}  ок

6-2-4 Построение робастных доверительных интервалов

\url{http://youtu.be/cqEg4bZNc-w}

от начала до 0:31 удалить

6-2-5 Тесты на автокорреляцию

\url{http://youtu.be/i6UMzh3EiTg}

0:07 убираем <<кафедра публичной политики>>



\end{document}