\documentclass[12pt,a4paper]{article}
\usepackage[utf8]{inputenc}
\usepackage[russian]{babel}
\usepackage[OT1]{fontenc}
\usepackage{amsmath}
\usepackage{amsfonts}
\usepackage{amssymb}
\usepackage{url}


\usepackage[left=2cm,right=2cm,top=2cm,bottom=2cm]{geometry}

\input{../emetrix_preamble}
\begin{document}

видео третьей главы:

3.1.1. Прогнозирование во множественной регрессии

\url{http://youtu.be/6x5FmRuTk08}

1:17 сейчас появляется надпись <<Доверительный интервал ...>> с формулой внизу. Они должны появиться только в 2:20

1:30 добавляем надпись <<Точность прогноза определяется шириной доверительного интервала>>

4:10 исправить на:

условное среднее, $E(y_i | X)$

ошибка прогноза условного среднего, $\hat{y}_i - E(y_i | X)$

дисперсия ошибки прогноза:

$Var(\hy_i - E(y_i | X)  | X )=Var(\hat{y}_i | X) = Var(\hat{\beta}_1 +\hat{\beta}_2 x_i + \hat{\beta}_3 z_i | X) $ 

5:04 исправить на:

конкретное наблюдение, $y_i$

ошибка прогноза, $\hy_i - y_i$

дисперсия ошибки прогноза:
\begin{multline} \nonumber
Var(\hat{y}_i - y_i | X)=Var(\hat{y}_i - E(y_i | X) - \varepsilon_i | X) = Var(\hat{y}_i - \varepsilon_i | X) = \\
Var(\hat{y}_i|X) + Var( \varepsilon_i | X) = Var(\hat{\beta}_1 +\hat{\beta}_2 x_i + \hat{\beta}_3 z_i | X) + Var(\epsilon_i | X)
\end{multline}

 (!сейчас опечатка в начале формулы! сейчас там написано $|X)|X)$, кроме этого в формуле всё ок )

6:25 опечатка --- пропущен пробел в <<предположении о нормальности>>

7:14 опечатка --- пропущен пробел в <<предположении о нормальности>>


3.1.2 Пример построения интервалов для прогнозов

\url{http://youtu.be/swWNng-RBho}

ок

3.1.3. Интерпретация коэффициента при логарифмировании

\url{http://youtu.be/NdsfDkPOqVI}

7:50 пропущен нижний индекс 2 у коэффициента бета, должно быть:

C ростом x на единицу у растет на $\beta_2$ единиц

7:57 пропущен нижний индекс 2 и окончание "ов"

C ростом x на один процент у растет на $\beta_2$ процентов

8:05 пропущен нижний индекс 2 у коэффициента бета, должно быть:

C ростом x на единицу у растет на $100\beta_2$ процентов

8:18 пропущен нижний индекс 2 и ошибка, должно быть:

C ростом x на один процент у растет на $\beta_2/100$ единиц


\newpage
3.1.4. Исправить название на <<Дамми-переменные. Разные зависимости для подвыборок>>

\url{http://youtu.be/wJj_oasiKgU}

0:28 немного исправить на: ($male_i$ --- синим цветом, тк это формула):

Например, переменная $male_i$, равная 1 для мужчин и 0 --- для женщин

4:02 перепутаны М и Ж, и лишний плюс в формуле, должно быть:

Для мужчин:

$wage_i=(\beta_1+\beta_4) + (\beta_2+\beta_5)exper_i+\beta_3 educ_i +\e_i$

4:16 перепутаны М и Ж, должно быть:

Для женщин:

(формула верно)

5:31 перепутаны М и Ж, должно быть:

Для мужчин:

5:39 перепутаны М и Ж, должно быть:

Для женщин:

6:34 перепутаны М и Ж, должно быть:

(формула)

Для мужчин: 

(формула)

Для женщин:

(формула)

7:45 сделать $season_i$ синим цветом 

8:03 в новом пункте немного переставить слова:

* Вводим три дамми-переменных

(четыре сезона минус один базовый)

8:08-8:24 сделать $vesna_i$, $leto_i$, $osen_i$ синим цветом 

8:30 очистить старые надписи и вывести такую табличку 

\begin{tabular}{ccccc}
\hline 
Наблюдение & Сезон & $vesna_i$ & $leto_i$ & $osen_i$  \\ 
\hline 
1 & Зима & 0 & 0 & 0 \\ 
2 & Весна & 1 & 0 & 0 \\ 
3 & Лето & 0 & 1 & 0 \\ 
4 & Осень & 0 & 0 & 1 \\ 
5 & Зима & 0 & 0 & 0 \\ 
$\vdots$ & $\vdots$ & $\vdots$ & $\vdots$ & $\vdots$ \\ 
\hline 
\end{tabular} 

12:00 оставляем прежний заголовок <<Частая ошибка!>> и оставляем подпись под ним

<<Включить дамми-переменные  на все значения факторной переменной и константу в регрессию>>

ниже появляется: (сейчас лишняя буква й в линейно)

<<Регрессоры линейно зависимы. Не существует единственных МНК оценок>> 

12:14 надпись появляется как сейчас, всё ок

12:22 надпись <<Синонимы...>> убираем полностью 



3.1.5. Проверка гипотезы о нескольких линейных ограничениях. (6 минут)

\url{http://youtu.be/NqJH0VGeEhU}

1:05 --- мигает pdf слайд, убираем

1:49 --- виден pdf слайд, убираем

2:57 --- виден pdf слайд, убираем

4:33 --- виден pdf слайд, убираем

4:48 сюда вставляем кусок от 3.1.6 начинающийся с 8:32 и до конца

\newpage
3.1.6. Пример проверки гипотезы о нескольких линейных ограничениях

\url{http://youtu.be/xludTSWApYM}

8:32---до конца отрезаем и вставляем в 3.1.5 в момент 4:48

3.1.7. добавить синию полосу с названием фрагмента в начале <<Гипотеза о незначимости регрессии>>

\url{http://youtu.be/fxnJk6bb7Yo}

0:24 появляется заголовок <<Гипотеза о незначимости регрессии>>

0:30 появляется формула

\[
y_i=\beta_1 + \beta_2 x_i + \beta_3 z_i + \beta_4 w_i + \beta_5 d_i +\beta_6 s_i + \e_i
\]

0:40 появляется еще одна формула с подписью ниже:

\[
H_0: \begin{cases}
\beta_2 = 0 \\
\beta_3 = 0 \\
\beta_4 = 0 \\
\vdots \\
\end{cases}
\]

Всего $(k-1)$ ограничение

1:12 очищаем экран (кроме заголовка) и пишем:

Общая формула:

\[
F=\frac{(RSS_R-RSS_{UR})/r}{RSS_{UR}/(n-k_{UR})} \sim F_{r,n-k_{UR}}
\]

Частный случай:

\[
F=\frac{ESS/(k-1)}{RSS/(n-k)} \sim F_{k-1,n-k}
\]



3.1.8. Пример проверки гипотезы о незначимости регрессии

\url{http://youtu.be/xbpA6Dji5mY}


3.1.9. Лишние и пропущенные регрессоры

\url{http://youtu.be/8O_36Dfeba8}

0:24 изменяем заголовок на <<Предпосылки:>> (убираем <<снова БСХС>>)

1:01 изменяем заголовок на <<Предположения на $\e_i$:>> (убираем БСХС)

1:27 перед формулой добавляем <<Условная некоррелированность>>, т.е. должно быть:

* Условная некоррелированность

$Cov(\e_i, \e_j | X)=0$ при $i \neq j$

1:34 изменяем заголовок на <<Предпосылки на регрессоры:>> (убираем БСХС)

2:05 изменяем заголовок на <<Асимптотические свойства (плюс новое):>> (убираем БСХС)

2:40 формулу $\hat{\sigma}^2=\frac{RSS}{n-k}$ удаляем полностью

3:03 изменяем заголовок на <<Свойства при нормальности (плюс новое):>> (убираем БСХС)

3:11 добавить двоеточие после слова <<Новое>>

3.1.10. Тест Рамсея

\url{http://youtu.be/51qvueLMChw}

2:20 исправить формулу, убрать крышку над первым $y_i$, должно быть:

$y_i = \beta_1 + \beta_2 x_i + \beta_3 z_i + \gamma_1 \hat{y}^2_i + \gamma_2 \hat{y}^3_i + \ldots + \gamma_p \hat{y}_i^{p+1} + \varepsilon_i$

\newpage
3.1.11. Простые показатели качества модели

\url{http://youtu.be/ZGlOncL0H4w}

1:32 немного поменять, чтобы вышло:

* $R^2=\frac{ESS}{TSS}=1-\frac{RSS}{TSS}$ 

растёт с добавление регрессоров, $R^2_{UR}>R^2_R$, 

$R^2=(sCorr(y,\hat{y}))^2$


3:52 Добавить строчку:

Штрафуем модель за большое $k$ и большую $RSS$

5:35 добавить в список пункт

* простые критерии качества 



3.2.1. Графики и переход к логарифмам

Диаграмма рассеяния для большого количества переменных, мозаичный график

\url{http://youtu.be/eZssYMz7Fgs}

5:36 - 6:36 (примерно) --- удалить из видео. По смыслу: там график выделился синим, и я пытался снять выделение, но никак не получалось, пока я его не перестроил. По-моему, хороший момент склейки --- когда я кликал на кнопочку zoom. От момента клика до момента клика можно вырезать.

3.2.2. Графики для качественных и количественных переменных 

(гистограммы разным цветом, фасетки)
\url{http://youtu.be/Ng-W5BqlpzE}

0:16 заменить название лекции на <<Проверка сложных гипотез и прогнозирование>>. Текущее название <<мультиколлинеарность>> относится к четвертой лекции.


3.2.3. Изменить название фрагмента на <<Оценивание моделей с дамми-переменными. Интерпретация>>

\url{http://youtu.be/Syab2Kc9Cs0}

0:16 заменить название лекции на <<Проверка сложных гипотез и прогнозирование>>. Текущее название <<мультиколлинеарность>> относится к четвертой лекции.


3.2.4. Изменить название фрагмента на <<Построение прогнозов. Доверительный и предиктивный интервал>>.

\url{http://youtu.be/c0wFwacaAV8}

0:16 заменить название лекции на <<Проверка сложных гипотез и прогнозирование>>. Текущее название <<мультиколлинеарность>> относится к четвертой лекции.


3.2.5. Изменить название фрагмента на <<Проверка гипотезы о линейных ограничениях. Графическое представление результатов>>

\url{http://youtu.be/Du0GNgkmyR8}

0:16 заменить название лекции на <<Проверка сложных гипотез и прогнозирование>>. Текущее название <<мультиколлинеарность>> относится к четвертой лекции.

3.2.6. Ловушка дамми-переменных. Информационные критерии. Тест Рамсея.

\url{http://youtu.be/FW3v22BDY98}

0:16 заменить название лекции на <<Проверка сложных гипотез и прогнозирование>>. Текущее название <<мультиколлинеарность>> относится к четвертой лекции.

3.2.7 Нано-исследование

\url{http://youtu.be/ySVk5QgHGfQ}

0:16 заменить название лекции на <<Проверка сложных гипотез и прогнозирование>>. Текущее название <<мультиколлинеарность>> относится к четвертой лекции.


\end{document}
