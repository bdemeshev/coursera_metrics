\documentclass[12pt,a4paper]{article}
\usepackage[utf8]{inputenc}
\usepackage[russian]{babel}
%\usepackage[OT1]{fontenc}
\usepackage{url}
\usepackage{amsmath}
\usepackage{amsfonts}
\usepackage{amssymb}
\usepackage[left=2cm,right=2cm,top=2cm,bottom=2cm]{geometry}
\begin{document}

4-1-1 Исправить название фрагмента  на <<Определение мультиколлинеарности>> (сейчас определие)

\url{http://youtu.be/9MhZ0UCb_Cc}

0:52 заменить появляющуюся матрицу на более длинную:

\[
\begin{pmatrix}
1 & 4 & 12 & 8 \\
1 & 3 & 3   & 3 \\
1 & 1 & 7   & 4 \\
1 & 2 & 4   & 3 \\
1 & 3 & 5   & 4 \\ 
\vdots & \vdots & \vdots & \vdots
\end{pmatrix}
\]

3:53 убрать полностью <<Практика>> и два подпункта

6:07 ошибка в начале формулы (пропущен квадрат), должно быть:

$se^2(\hat{\beta}_j)=\frac{\hat{\sigma}^2}{RSS_j}=\frac{\hat{\sigma}^2}{TSS_j\cdot (1-R^2_j)}=
\frac{1}{1-R^2_j}\frac{\hat{\sigma}^2}{TSS_j}$

9:57 ошибка в начале формулы (пропущен квадрат), должно быть:

$se^2(\hat{\beta}_j)=VIF_j \cdot \frac{\hat{\sigma}^2}{TSS_j}$

10:19 после пункта <<Выборочные корреляции...>> добавить формулу

$sCorr(x,z)=\frac{\sum (x_i - \bar{x}) (z_i - \bar{z}) }{\sqrt{sVar(x) \cdot sVar(z)}}$

10:32 ничего нового не появляется

10:36 старый текст очищается, появляется новый (это исправленный вариант того, что было в 10:32):

Некоторые источники считают признаком мультиколлинеарности:

* $VIF_j > 10$

* $sCorr(x,z)>0.9$

10:57 фрагмент видео обрывается неожиданно на незаконченной фразе

4-1-2 Что поделать с мультиколлинеарностью?

\url{http://youtu.be/yXQGrWGVpoo}

1:21 Исправить второй пункт на <<Уменьшить дисперсию оценок, пожертвовав их несмещенностью>>

1:28 Исправить третий пункт на <<Мечта: уменьшить дисперсию оценок, используя больше наблюдений>>

до 2:05 done

2:05 изменяем заголовок слайда на <<Жертвуем несмещнностью, чтобы снизить дисперсию>>

2:10 надпись <<Модель зависит от всех регрессоров>> убираем полностью

2:15 Исправить надпись на 

* Выкинуть часть регрессоров. 

Жертвуем: знанием выкидываемых коэффициентов, несмещенностью оставшихся коэффициентов.

2:23 Исправить надпись на

* Использовать МНК со штрафом

Жертвуем: несмещенностью коэффициентов

2:24 повторную надпись <<Жертвуем несмещенностью!>> убираем полностью

до 2:51 done


  \url{http://youtu.be/fwqs8mPEZFc}

\url{http://youtu.be/6LnUUI0Iotw}

\url{http://youtu.be/oRYyXjZJipo}

 \url{http://youtu.be/-UcSI47ITfU}

\url{http://youtu.be/9P4pCJN9ENw}

\url{http://youtu.be/ixdKGZJeSt0}

\url{http://youtu.be/dAJu-BUyvks}

\url{http://youtu.be/g83D9xxOWzE}

\end{document}